\chapter{总结与展望}


以上就是笔者在使用Latex总结的一些基本经验,可能也有一些不恰当的地方,欢迎评论和指正。Latex具体的一些操作实在有太多太多,大家也可以参考网上其他教程进行相关的高级操作,这里就不再赘述啦。

(最后一章为“总结与展望”。包括对整个论文主要成果的总结,本研究的创造性成果或创新点理论,对应用前景和社会、经济价值等的预测和评价,并指出今后进行进一步研究工作的展望与设想。写作时删除此段说明。)

这里引用两个参考文献\cite{2001Applying}\cite{2004PSO_ZhangLibiao}。

还可以这样引用\cite{2001Applying,2021A}。

或者说这样引用\cite{2001Applying,2004PSO_ZhangLibiao,2021A}

引用作者是这样引用\citet{2001Applying},\citet{2004PSO_ZhangLibiao}

或者是这样引用\citep{2001Applying}

注意参考文件样式一共有两个标准

\begin{itemize}
    \item GBT7714-2005
    \item GBT7714-2015
\end{itemize}

其中,numerical表示的是按照数字顺序排序的,author-year表示的是作者首字母排序,优先中文排序。
在本文中默认选择GBT7714-2015-author-year。

在提交的论文中,有些同学出现了以下一些参考文献的错误,注意一些常见问题:
\begin{enumerate}
    \item 参考文献的写法不够统一,例如,会议信息有的写了 Proceedings,有的没写。我的建议是按照ACM官方上的bib格式进行一些信息补充再生成对应的PDF,把会议内容一定要写全,例如下面这个例子\citet{zheng-etal-2017-joint}(参考bib文件);
    \item 关于Arxiv上的论文参考,有些评论意见出现有缺少期卷、页码的情况,注意写成对应的格式,参考例子\citet{wang2023instructuie}(参考bib文件)。笔者认为arxiv文件属于R,A,Z/OL类别,不过有需要的同学可以自行酌情修改;
    \item 参考文献可以去网站DBLP\footnote{\url{https://dblp.org}}上进行检索,可以引入较为完整的bib格式,但是需要注意字段的修改。
\end{enumerate}
